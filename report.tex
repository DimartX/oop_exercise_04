\documentclass[12pt]{article}

\usepackage{fullpage}
\usepackage{multicol,multirow}
\usepackage{tabularx}
\usepackage{ulem}
\usepackage[utf8]{inputenc}
\usepackage[russian]{babel}
\usepackage{minted}

\usepackage{color} %% это для отображения цвета в коде
\usepackage{listings} %% собственно, это и есть пакет listings

\lstset{ %
language=C,                 % выбор языка для подсветки (здесь это С)
basicstyle=\small\sffamily, % размер и начертание шрифта для подсветки кода
numbers=left,               % где поставить нумерацию строк (слева\справа)
%numberstyle=\tiny,           % размер шрифта для номеров строк
stepnumber=1,                   % размер шага между двумя номерами строк
numbersep=5pt,                % как далеко отстоят номера строк от подсвечиваемого кода
backgroundcolor=\color{white}, % цвет фона подсветки - используем \usepackage{color}
showspaces=false,            % показывать или нет пробелы специальными отступами
showstringspaces=false,      % показывать или нет пробелы в строках
showtabs=false,             % показывать или нет табуляцию в строках
frame=single,              % рисовать рамку вокруг кода
tabsize=2,                 % размер табуляции по умолчанию равен 2 пробелам
captionpos=t,              % позиция заголовка вверху [t] или внизу [b] 
breaklines=true,           % автоматически переносить строки (да\нет)
breakatwhitespace=false, % переносить строки только если есть пробел
escapeinside={\%*}{*)}   % если нужно добавить комментарии в коде
}

% Оригиналный шаблон: http://k806.ru/dalabs/da-report-template-2012.tex

\begin{document}
\begin{titlepage}
\begin{center}
\textbf{МИНИСТЕРСТВО ОБРАЗОВАНИЯ И НАУКИ РОССИЙСОЙ ФЕДЕРАЦИИ
\medskip
МОСКОВСКИЙ АВЦИАЦИОННЫЙ ИНСТИТУТ
(НАЦИОНАЛЬНЫЙ ИССЛЕДОВАТЬЕЛЬСКИЙ УНИВЕРСТИТЕТ)
\vfill\vfill
{\Huge ЛАБОРАТОРНАЯ РАБОТА №4} 
по курсу объектно-ориентированное программирование
I семестр, 2019/20 уч. год}
\end{center}
\vfill

Студент \uline{\it {Артемьев Дмитрий Иванович, группа М8О-206Б-18}\hfill}

Преподаватель \uline{\it {Журавлёв Андрей Андреевич}\hfill}

\vfill
\end{titlepage}

\subsection*{Условие}

Задание: \
Вариант 1: Треугольник, Квадрат, Прямоугольник.\
Разработать шаблоны классов согласно варианту задания. Параметром шаблона должен являться скалярный тип данных, задающий тип данных для оси координат. Классы должны иметь публичные поля. 
\begin{enumerate}
\item Вычисление геометрического центра фигуры;
\item Вывод в стандартный поток вывода std::cout координат вершин фигуры;
\item Вычисление площади фигуры;
\end{enumerate}

Создать программу, которая позволяет:
\begin{enumerate}
\item Вводить из стандартного ввода std::cin фигуры, согласно варианту задания (как в виде класса, так и в виде std::tuple).
\item Вызывать для неё шаблонные функции(1-3).
\end{enumerate}

\subsection*{Описание программы}

Исходный код лежит в 11 файлах:
\begin{enumerate}
\item src/main.cpp: основная программа, взаимодействие с пользователем посредством комманд из меню
\item include/figure.hpp:    описание шаблона класса фигур
\item include/point.hpp:     описание шаблона класса точки 
\item include/triangle.hpp:  описание шаблона класса треугольника, наследующегося от figures
\item include/rectangle.hpp: описание шаблона класса прямоугольника, наследующегося от figures
\item include/square.hpp:    описание шаблона класса квадрата, наследующегося от rectangle
\item include/template.hpp:  функции для чтения/вывода tuple
\item include/functions.hpp: функции для нахождения площади, центра фигуры и вывода её координат на экран
\end{enumerate}

\subsection*{Дневник отладки}

Долгие попытки прикрутить фичи C++17.

\subsection*{Недочёты}

Нечитабельный код.

\subsection*{Выводы}

Научился использовать шаблоны в C++, изучил коллекцию tuple и некоторые фишки C++17.

\vfill

\subsection*{Исходный код}

{\Huge main.cpp}
\inputminted
    {C++}{src/main.cpp}
    \pagebreak
    
{\Huge figure.hpp}
\inputminted
    {C++}{include/figure.hpp}
    \pagebreak

{\Huge point.hpp}
\inputminted
    {C++}{include/point.hpp}
    \pagebreak

{\Huge triangle.hpp}
\inputminted
    {C++}{include/triangle.hpp}
    \pagebreak

{\Huge rectangle.hpp}
\inputminted
    {C++}{include/rectangle.hpp}
    \pagebreak

{\Huge square.hpp}
\inputminted
    {C++}{include/square.hpp}
    \pagebreak

{\Huge template.hpp}
\inputminted
    {C++}{include/template.hpp}
    \pagebreak

{\Huge functions.hpp}
\inputminted
    {C++}{include/functions.hpp}
    \pagebreak
    
\end{document}
